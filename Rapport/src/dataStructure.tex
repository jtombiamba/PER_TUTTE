Dans l'implémention de l'algorithme nous avons défini notre propre structure de données sur laquelle nous avons deroulé l'algorithme. Des moules nous permettent de passer d'un graphe tulip à notre structure et aussi de recuperer des informations de notre structure pour les insérer dans un graphe tulip de sorte à ce que notre structure soit uns structure temporaire de stockage d'informations sur les noeuds d'un graphe passé en paramètre de notre programme. 

\section{Aim}
Il est très couteux de manipuler les graphes tulip et surtout nous n'avons pas besoin de toutes les propriétés des noeuds pour le déroulement de l'algo de tute. En plus il fallait une structure légère et adaptée à l'algorithme de tutte danas la mesure où nous sommes en quête de performance. Ci-dessous quelques raisons qui nous ont conduit à la mise en place d'une nouvelle structure de données.
\begin{enumerate}
\item La propriétés nous nous avons seulement bésoin de savoir si un sommet est au bord pour le considérer comme fixe pendant le déroulement de l'alo de tutte.
\item
\item
\item
\end{enumerate}  

\section{Structure's contents}

Dans l'implémentation de notre structure pour éviter les charges en mémoire et faciliter sa anipuler l'information la plus minimale possible pour l'exécution de l'algorithme de tutte sur un graphe. Pour ce faire nous définissons une classe qui va contenir les diifértentes dont nous avons bésoins sur un noeud (les attibuts) et toutes les opérations dont nous aurons bésoin d'effectuer sur un noeud(les méthodes).

\begin{lstlisting}
class MyNode {
 private:
  node n;
  bool mobile;
  Coord coord;  
  vector<MyNode *> voisin;

 public:
  MyNode();
  MyNode(const node n, const Coord coord);
  MyNode(const node n, const bool mob, const Coord coord);
  ~MyNode();
  
  const node getNode() const;
  bool getMobile() const;
  void setMobile(const bool b);
  const Coord getCoord() const;
  void setCoord(const Coord &);
  vector<MyNode *> * getVoisin();
  vector<MyNode *> getVoisin() const;
};
\end{lstlisting}

\subsection{The vertex's attributs needed}


\subsection{The operations on a vertex}
