\chapter{Tutte's algorithm}

This part of text comes from the article~\cite{pa}.

In this paper, we will use basic graph theory terminology, see for
example~\cite{pb}. Let $G=(V,E)$ be a planar graph. A mapping $\Gamma$
of $G$ into the plane is a function $\Gamma : V \cup E \to P(\mathbb{R}^2)$
which maps a vertex $v \in V$ to a point in $\mathbb{R}^2$ and an edge $e =
uv \in E$ to the straight line segment joining $\Gamma(u)$ and
$\Gamma(v)$.  A mapping is an embedding if distinct vertices are mapped to
distinct points, and the open segment of each edge does not intersect any
other open segment of an edge or a vertex.

In 1963, Tutte~\cite{pc} gave a way to build embeddings of any planar,
3-connected graph $G=~(V,E)$. Let $C$ be a cycle whose vertices are the
vertices of a face of G in some (not necessarily straight-line) embedding
of $G$. Let $\Gamma$ be a mapping of $G$ into the plane, satisfying the
conditions:

\begin{itemize}

\item the set Ve of the vertices of the cycle C is mapped to the vertices of a strictly
convex polygon Q, in such a way that the order of the points is respected;

\item each vertex in $V_i = V \ V_e$ is a barycenter with positive coefficients of
its adjacent vertices (Tutte assumed all coefficients to be equal to 1, but
the proof extends without changes to this case). In other words, the images
v of the vertices v under $\Gamma$ are obtained by solving a linear system
(S): for each $u \in V_{i, v|uv \in E} \lambda_{uv} (u - v) = 0$, where the
$\lambda_{uv}$ are positive reals. It can be shown that the system (S) admits
a unique solution.

\end{itemize}

\begin{theo} \label{theo:box}
(Tutte’s Theorem) $\Gamma$ is an embedding of G into the plane, with
strictly convex interior faces.
\end{theo}


% \begin{thebibliography}{99}

% \bibitem{pa} E. Colin de Verdière, M. Pocchiola, and G. Vegter. Tutte's Barycenter Method applied to Isotopies. \emph{Computational Geometry: Theory and Applications, 26}, 81–97, 2003.

% \bibitem{pb} B. Bollob's. Modern graph theory, \emph{volume 184 of Graduate Texts in Mathematics}. Springer-Verlag, 1998.

% \bibitem{pc} W. T. Tutte. How to draw a graph. \emph{Proceedings of the London Mathematical Society}, 13:743–768, 1963.

% \end{thebibliography}


